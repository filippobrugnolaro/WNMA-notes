\section{Radio Frequency}
Most wireless communications are based on this technology.
\subsection{Properties}
Here is some characteristics and properties of radio freuquency:
\begin{itemize}
    \item Antenna:
    \begin{itemize}
        \item[$\rightarrow$] it has high frequency alternate current
        $\Rightarrow$ generates electromagnetic\\energy
        \item[$\rightarrow$] it converts wired current to radio
        frequency and viceversa
        \item[$\rightarrow$] it can produce radio frequency
        with different frequency/amplitude\\
        $\rightarrow$ as signal propagates $\Rightarrow$ it becomes
        weaker and weaker
    \end{itemize}
    \item Frequency $\rightarrow$ it is the number of waves in a second:
    \begin{itemize}
        \item[$\rightarrow$] there is a wireless spectrum
        (regulated and free areas)
        \item[$\rightarrow$] wavelenght $= \frac{\text{c}}{\text{freq}}$
        $\Rightarrow$ distance between spikes\\
        $\rightarrow$ it gives antenna's recommended lenght\\
        $\rightarrow$ it works better if size is $\frac{1}{2^n}$
        lenght of wavelenght
    \end{itemize}
    \item Amplitude:
    \begin{itemize}
        \item[$\rightarrow$] higher amplitude signals $\Rightarrow$ it goes
        further
        \item[$\rightarrow$] transmission power $= \frac{\text{energy}}{\text{time}}
        \rightarrow \frac{\text{joule}}{\text{s}}$
    \end{itemize}
    \item Coverage:
    \begin{itemize}
        \item[$\rightarrow$] as distance grows $\Rightarrow$ signal becomes
        weaker in an exponetial decline\\
        $\rightarrow$ you can detect a weak signal $\rightarrow$ but you can't
        really use it\\
        \hspace*{5cm}(weak for exchanging messages)
        \item[$\rightarrow$] problems:
        \begin{itemize}
            \item obstacles $\rightarrow$ can reflect or absorbe waves
            \\$\rightarrow$ it depends on material and frequency
            \\$\rightarrow$ rules of thumbs 
            \begin{itemize}
                \item high frequency $\rightarrow$ short distances, more affected
                by obstacles
                \item low frequency $\rightarrow$ long distances, less affected
                by obstacles
            \end{itemize}
            \item phase shifting $\rightarrow$ positve/negative aspects
            $\rightarrow$ early/late wavefront\\$\rightarrow$ signals can be null
            and overlap each other
        \end{itemize}
        \item[$\rightarrow$] polarisation $\rightarrow$ phisical orientation of
        antenna
        \begin{itemize}
            \item radio frequency is made up of 2 perpendicular fields\\
            (electric/magnetic)
            $\Rightarrow$ the presence of:
            \begin{itemize}
                \item Horizontal polarisation $\rightarrow$ electric field
                parallel to ground
                \item Vertical polarisation $\rightarrow$ electric field
                perpendicular to ground
                $\rightarrow$ if 2 antennas are perpendicular to ground
                $\Rightarrow$ better transmission
            \end{itemize}
        \end{itemize}
    \end{itemize}
\end{itemize}

\subsection{Wireless Transmission}

It happens through elettromagnetic waves. There is a dependency on\\amplitude,
frequency and phase values $\rightarrow$ each combination produces a new signal

Characteristics:
\begin{itemize}
    \item Range:
    \begin{itemize}
        \item[$\rightarrow$] Transmission: communication possible, low error rate
        \item[$\rightarrow$] Detection: detection of signal, no exchanging
        messages
        \item[$\rightarrow$] Interference: no detection for too much noise
        depending from many \\factors (distance, environment\dots)
    \end{itemize}
    Detection requires more energy than communication
    \item Propagation:
    \begin{itemize}
        \item[$\rightarrow$] it is at the light speed in free spaces
        \item[$\rightarrow$] receiving power dipends from distance between
        sender/receiver\\
        rp = $\frac{1}{\text{d}^2}$ $\rightarrow$ rp influenced by:
        \begin{itemize}
            \item fading (dependent on frequency)
            \item shadowing (obstacles)
            \item reflection (large obstacles)
            \item refraction (density of obstacles)
            \item scattering (small obstacles)
            \item diffraction (at edges)
        \end{itemize}
        \item[$\rightarrow$] signal can follow different paths due to refraction,
        scattering, diffraction. So there is:
        \begin{itemize}
            \item Time dispersion $\rightarrow$ signal is dispersed over time
            \item Phase shifting $\rightarrow$ signal is distorted
        \end{itemize}
    \end{itemize}
    \item Power measurement
    \begin{itemize}
        \item[$\rightarrow$] It is the Decibel (dB) $\rightarrow$ expression
        power loss
        \item[$\rightarrow$] It is more pratical to use logarithmic decay
        $\rightarrow$ easy calculations
        \item[$\rightarrow$] Decibel measures the logarithmic relative strenght
        between 2 signals
        \item[$\rightarrow$] Values of power measuements:
        \begin{itemize}
            \item positive $\rightarrow$ power gain
            \item negative $\rightarrow$ power loss
        \end{itemize}
    \end{itemize}
\end{itemize}

\subsection{Antennas}

Characteristics:
\vspace{-0.1cm}\begin{itemize}
    \item it converts electrical energy in radio
    frequency waves (transmission)\\and viceversa (reception)
    \item its size $\rightarrow$ depends on radio
    frequency of transimission/reception
    \item its shape $\rightarrow$ depends on radio
    frequency radiation pattern
    \item position important to have max coverage
\end{itemize}

There are different types of antennas:
\begin{itemize}
    \item Omnidirectional antennas
    \item Semi-directional antennas
    \item Highly-directional antennas
    \item Sectorised-directional antennas
\end{itemize} 
\subsubsection{Omnidirectional antennas}

Characteristics:
\begin{itemize}
    \item radio frequency power is equally distributed in all direction around
    Y-axis
    \item used when:
    \begin{itemize}
        \vspace{-0.1cm}\item[$\rightarrow$] need of uniform radio coverage
        \vspace{-0.1cm}\item[$\rightarrow$] point-to-multipoint connections (star topology)
    \end{itemize}
    \item Tilt $\rightarrow$ it is degree of inclination of antenna with respect to Y-axis
    \item Example of dipole antenna
    \begin{itemize}
        \item[$\rightarrow$] passive gain due to concentration of radiations
        \item[$\rightarrow$] active gain obtained with power amplifiers
        \item[$\rightarrow$] signal is weak near the dipole
        \item[$\rightarrow$] there is also:
        \begin{itemize}
            \setlength\itemsep{0.0000001em}
            \vspace{-0.1cm}\item low gain $\rightarrow$ high signal near antenna, low far
            \vspace{-0.1cm}\item high gain $\rightarrow$ low signal near antenna, high far
        \end{itemize}
    \end{itemize}
\end{itemize} 
\subsubsection{Semi-directional antennas}

Characteristics:
\begin{itemize}
    \item radio frequency power is equally distributed only on $\frac{1}{2}$ direction\\
    (also few goes behind that direction)
    \item Types:
    \begin{itemize}
        \item[$\rightarrow$] Patch $\rightarrow$ flat antennas mounted on walls
        \item[$\rightarrow$] Panel $\rightarrow$ flat antennas mounted on walls
        \item[$\rightarrow$] Yagi $\rightarrow$ rod with tines sticking out 
    \end{itemize}
\end{itemize}
\subsubsection{Highly-directional antennas}

Characteristics:
\begin{itemize}
    \item radio frequency power is distributed on a specific direction and antenna could be as:
    \begin{itemize}
        \item[$\rightarrow$] parabolic dish
        \item[$\rightarrow$] grid
    \end{itemize}
    \item it is used for long distances $\rightarrow$ point-to-point link
    \item there is what is called LoS (Line of Sight):
    \begin{itemize}
        \item[$\rightarrow$] straight line between sender and receiver
        \item[$\rightarrow$] needs no obstruction
    \end{itemize}
    \item there is also the Freshnel Zone:
    \begin{itemize}
        \item[$\rightarrow$] it is an area which is centered on LoS axis
        \item[$\rightarrow$] most additive radio frequency signal is concentrated here
        \item[$\rightarrow$] there is the need of no obstacles\\ (useless increasing power if Freshnel Zone is not free)
        \item[$\rightarrow$] it depends on distance and frequency\\ $\Rightarrow$ there is no dependency from type, degree, gain of antennas
    \end{itemize}
\end{itemize}
\subsubsection{Sectorised-directional antennas}

Characteristics:
\begin{itemize}
    \item there are multiple antennas $\rightarrow$ each one points to a direction
    \item it is applied the space multiplexing (channel reuse)\\ $\Rightarrow$ assigned the same frequency for antennas which do not collide each others 
\end{itemize}