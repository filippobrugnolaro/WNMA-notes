\section{802.11 Standard}
There are a lot of different standards:
\begin{itemize}
    \item to enhance QoS, interoperability with existing technologies
    \item to enhance speed, spectrum occupied, frequency, \dots
    \item for particular use cases (VANET, \dots)
\end{itemize}
Recent standards must be retrocompatible and some of them are:
\begin{itemize}
    \item n $\rightarrow$ Wi-Fi 4 $\rightarrow$ new denomination by Wi-Fi Alliance
    \item ac $\rightarrow$ Wi-Fi 5 $\rightarrow$ MIMO, > bandwidth than n$\rightarrow$
    up to 500 Mbps
    \item ax $\rightarrow$ Wi-Fi 6 $\rightarrow$ > frequency range, < latency, uplink multiuser MIMO
    \item be $\rightarrow$ Wi-Fi 7 $\rightarrow$ 3 main frequency, coordinated multiuser MIMO
    \item mc $\rightarrow$ indoor localization $\rightarrow$ using Wi-Fi Round Trip Time (Wi-Fi RTT) 
\end{itemize}
\subsection{802.11e}
Characteristics:
\begin{itemize}
    \item standard with a QoS $\rightarrow$ technology must provide specific quality
    \item Exchanges system:
    \begin{itemize}
        \item[$\rightarrow$] Before
        \begin{itemize}
            \item DCF (Distributed Coordination Function)\\
            $\Rightarrow$ classic approach
            \item PCF (Point Coordination Function)\\
            $\Rightarrow$ $\approx$ polling 
        \end{itemize}
        \item[$\rightarrow$] Now
        \begin{itemize}
            \item EDCA (Enhanced Distributed Channel Access $\rightarrow$ Priority Scheme)
            \item HCCA (Hybrid Controlled Channel Access $\rightarrow$ Parameterized QoS Scheme)
        \end{itemize}
    \end{itemize}
\end{itemize}
\subsubsection{DCF (Distributed Coordination Function)}
Characteristics:
\begin{itemize}
    \item it is the basic access method
    \item it uses CSMA/CA $\rightarrow$ if channel is:
    \begin{itemize}
        \item[$\rightarrow$] free $\Rightarrow$ it waits DIFS + CW and then transmit if still free
        \item[$\rightarrow$] busy $\Rightarrow$ so:
        \begin{itemize}
            \item random exponential backoff a number of slots (from 32 to 1024)
            \item count down slots as long as Medium is not busy
            \item count down $=$ 0 $\rightarrow$ if packet fails (collision)
            $\Rightarrow$ exponential back-off again with increased random window
        \end{itemize}
    \end{itemize}
    \item Contention Windows (CW):
    \begin{itemize}
        \item[$\rightarrow$] it is a random number selected in the interval [0,CW]
        \item[$\rightarrow$] small value for CW $\Rightarrow$ so:
        \begin{itemize}
            \item < wasted idle slots time
            \item > number of collisions with multiple senders 
        \end{itemize}
        \item[$\rightarrow$] optimal CW, knowing number of nodes and size of packets, is obtained:
        \begin{itemize}
            \item minimizing time wastage $\rightarrow$ collisions/empty simultaneous
            \item hard to implement $\rightarrow$ number of nodes hard to estimate + dynamic
        \end{itemize}
        \item[$\rightarrow$] adaptive CW
        \item[] \begin{itemize}
            \item idea:
            \begin{itemize}
                \item start with CW $=$ 31
                \item no CTS/ACK $\Rightarrow$ increase size to 2 $\cdot$ CW $+$ 1
                \item successful transmission $\Rightarrow$ reset CW $=$ 31
            \end{itemize}
            \item adaptive scheme is unfair\\[0.15cm]
            $\rightarrow$ luckily nodes can transmit several packets while other are waiting
            $\Rightarrow$ unlucky nodes continue doubling CW $\rightarrow$ no transmission
            \item Adaptive CW doesn't provide QoS
        \end{itemize}
    \end{itemize}
\end{itemize}
\subsubsection{PCF (Point Coordination Function)}
Characteristics:
\begin{itemize}
    \item priority system centrally controlled $\rightarrow$ Point Coordinator (usually AP)
    \item after each beacon $\rightarrow$ there is Contention Period and Contention Free Access\\[0.15cm]
    $\rightarrow$ fixed length of time after a beacon $\rightarrow$ synchronized to beacon's intervals
    \item it uses PIFS to keep control $\rightarrow$ < than DCF
    \item Point Coordinator keeps list of station eligible for polling
    \item it is not used in practice:
    \begin{itemize}
        \item[$\rightarrow$] not compatible with certain traffic flows (voice, video, \dots)\\[0.15cm]
        $\Rightarrow$ no bandwidth reservation / no traffic characterisation
        \item[$\rightarrow$] no back-to-back transmission of packets
    \end{itemize}
\end{itemize}
\subsubsection{ECDA (Enhanced Distributed Channel Access)}
Characteristics:
\begin{itemize}
    \item there are 4 Access Categories (AC) + 8 Traffic Classes (TC)
    \item Max Service Data Unit (MSUD) delivered through multiple backoffs\\[0.15cm]
    $\rightarrow$ with one station using specific AC parameters
    \item each AC starts to backoff $\rightarrow$ after detecting channel being idle for AIFS
    \item after waiting AIFS $\rightarrow$ each backoff sets number from interval [1, CW$+$1]
    \item CW$_{new}$[AC] $=$ ((CW$_{old}$[TC]$+$1) $\cdot$ PF)
    \item CA is realised with different parameters:
    \begin{itemize}
        \item[$\rightarrow$] AIFS[AC] $=$ Arbitration Inter-Frame Space
        \item[$\rightarrow$] CW$_{min}$[AC] (or CW$_{max}$[AC]) $=$ Contention Window min (or max)
        \item[$\rightarrow$] PF[AC] $=$ Persistence Factor
    \end{itemize}
    \item similar to DCF with priorities:
    \begin{enumerate}
        \item voice
        \item video
        \item best effort
        \item background
    \end{enumerate}
    $\Rightarrow$ bursting (more packets at once) possible only for video/voice\\[0.15cm]
    $\rightarrow$ different min/max backoff (hard to find) $\Rightarrow$ gives advantage to some type of traffic ($\Rightarrow$ > CW after backoff)
    \item Pros:
    \begin{itemize}
        \item[$\rightarrow$] voice/video are prioritised over data
        \item[$\rightarrow$] it works well if network is lightly loaded
        \item[$\rightarrow$] no stream setup required
        \item[$\rightarrow$] high power save respect to legacy version 
    \end{itemize}
    \item Cons:
    \item \begin{itemize}
        \item[$\rightarrow$] it is not fair $\rightarrow$ lower priority still need to be transmitted
        \item[$\rightarrow$] there can be competition of stream of the same category\\
        $\rightarrow$ difficult to guarantee high bandwidth, low latency \dots
        \item[$\rightarrow$] it relies on STA/AP to control priorities $\rightarrow$ QoS variation occurs
    \end{itemize}
    \item Admission Control:
    \begin{itemize}
        \item[$\rightarrow$] it introduces to guarantee QoS to ECDA
        \item[$\rightarrow$] it limit admission to an Access Category (voice and video)
        \begin{itemize}
            \item it limits the latency of QoS streams
            \item it prevents too many streams such that bandwidth can't handle them
        \end{itemize}
        \item[$\rightarrow$] Flow:
        \begin{enumerate}
            \addtolength{\itemindent}{-0.3cm}
            \item AP advertises ACM bit in beacon\\
           \hspace*{-0.3cm}$\rightarrow$ indicating if admission control is required for any Access Category
           \item STA sends AddTS\footAddTS Request Action Frame to AP $\rightarrow$ which includes a
           \hspace*{-0.3cm}TSPEC:
           \begin{itemize}
            \addtolength{\itemindent}{-0.3cm}
            \item Nominal MSUD size
            \item Mean data rate
            \item Min PHY rate
            \item Surplus bandwidth allowance (SBA)
           \end{itemize}
           \item AP runs admission control algorithm and sends back the response to
           \hspace*{-0.3cm}STA using AddTS Response Action Frame
        \end{enumerate}
        $\Rightarrow$ STA checks used time\\
        $\rightarrow$ if used\_time > medium\_time $\Rightarrow$ STA must use AC's ECDA parameters
    \end{itemize}
    \item Pros:
    \begin{itemize}
        \item[$\rightarrow$] ECDA improvement:
        \begin{itemize}
            \item it contains high priority streams
            \item it protects streams in progress 
        \end{itemize}
        \item[$\rightarrow$] TSPEC can specify traffic specification desired
    \end{itemize}
    \item[$\rightarrow$] Cons:
    \begin{itemize}
        \item[$\rightarrow$] bandwidth efficiency not optimum
        \item[$\rightarrow$] fairness not guaranteed for streams of the same category
    \end{itemize} 
\end{itemize}
\subsubsection{HCCA}
Characteristics:
\begin{itemize}
    \item it is extension of PCF with Contention Free Period
    \item there is an hybrid controller (HC) which initiates HCCA and PCF:
    \begin{itemize}
        \item[$\rightarrow$] it provides CF-poll to station to provide transmission opportunities
        \item[$\rightarrow$] it specifies start$_{time}$ and max$_{dur}$ $\rightarrow$ other stations can't access to the medium
        \item[$\rightarrow$] STA transmits within SIFS and then using PIFS periods
        \item[$\rightarrow$] if no transmission after PIFS $\Rightarrow$ so:
        \begin{itemize}
            \item it issues new transmission opportunity
            \item it issues end of CFP
        \end{itemize} 
        \item[$\rightarrow$] CFPs can be synchronized to individual source of traffic intervals
    \end{itemize} 
    \item it allows for CFPs being initiated at any time during CP\\
    $\rightarrow$ called CAP (Control Access Phase):
    \begin{itemize}
        \item[$\rightarrow$] it is initiated by AP (if it wants to send a frame or receive one from STA) $\rightarrow$ contention free manner guarantee
        \item[$\rightarrow$] during a CAP $\rightarrow$ HC (i.e. AP) controls the access to the medium
        \item[$\rightarrow$] during CP $\rightarrow$ all stations function in ECDA
    \end{itemize}
    \item HC is not limited to per-station querying $\rightarrow$ it can provide per-session service
    \item Advantages:
    \begin{itemize}
        \item efficient use of bandwidth $\rightarrow$ CFP used, channel free as soon as\\packet is sent for that transmission opportunity
        \item latency guarantee $\rightarrow$ grant transmission opportunities as required by TSPEC
        \item bandwidth guarantee $\rightarrow$more efficient use of ECDA $\rightarrow$ no limited backoff
        \item all STA/AP hearing QoS poll will obey to transmission opportunity
        \item ACK from QSTA should include duration field $\rightarrow$ outstanding\\transmission time
        $\Rightarrow$ extend range of CFP to other networks
    \end{itemize}
    \item Disadvantages:
    \begin{itemize}
        \item complex scheduler $\rightarrow$ more complexity in general
        \item overlapping HCCA networks have transmission opportunity problems
    \end{itemize}
\end{itemize}
\subsection{802.11n}
Characteristics:
\begin{itemize}
    \item it enables new consumers (enterprise application)\\
    $\rightarrow$ thanks to more bandwidth, ranging, throughput, \dots $\rightarrow$ QoS
    \item High data rate (up to 300 Mbps with 2 MIMO devices)
    \item legacy support
    \item Components:
    \begin{itemize}
        \item[$\rightarrow$] PHY (2.4/5 GHz):
        \begin{itemize}
            \item support of OFDM modulation
            \item MIMO with spacial multiplexing
            \item higher throughput $\rightarrow$ combination of adjacent channels
        \end{itemize}
        \item[$\rightarrow$] MAC:
        \begin{itemize}
            \item aggregation methods $\rightarrow$ pack smaller packets into single MPUD
            \item combined ACK $\rightarrow$ no need to ACK every packet, once every few time
        \end{itemize}
    \end{itemize}
    \item MIMO (Multiple Input Multiple Output):
    \begin{itemize}
        \item[$\rightarrow$] it transmits and receives with multiple radios simultaneously on the\\same spectrum
        \item[$\rightarrow$] outages are reduced $\rightarrow$ using INFO from different antennas
        \item[$\rightarrow$] multiple power amplifiers $\Rightarrow$ more transmission power
        \item[$\rightarrow$] higher throughput
        \item[$\rightarrow$] interferences limited with some techniques 
    \end{itemize}
    \item SISO (Single Input Single Output):
    \begin{itemize}
        \item[$\rightarrow$] simple and low cost
        \item[$\rightarrow$] outages can occur if antennas fall into null $\rightarrow$ need to switch antenna
        \item[$\rightarrow$] energy wastage by sending in all direction $\rightarrow$ possibly more interferences
        \item[$\rightarrow$] more sensitive to interferences
        \item[$\rightarrow$] Output power limited by single power amplifier
    \end{itemize}
\end{itemize}