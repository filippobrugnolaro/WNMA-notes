\section{Introduction}
\subsection{Wireless Development}

\textbf{Present}\\ it is constantly growing
due to higher use of laptops or devices which can
connect to internet. This implied an
important growth of WiFi and n-G (3G, 4G, 5G)
technologies also thanks to the emerging of
apps with both low and high data demand.
Smartphones open to new wireless scenarios such as
AR, VR, MR, tele-presence\dots\\Other topics are
Tactile Internet (combination of low latency, high
availability, reliability and security) and Web
Squared (integration of web 2.0 with technologies of
sensing).

\textbf{Future}\\ it is based on ubiquitous
communication among people and devices. So this implies
to take into account some requirements such as 
bandwidth, delay, energy and connectivity.

\textbf{Challenges}
\begin{itemize}
    \item Wireless channels are a difficult and
    capacity-limited broadcast communications medium
    (with respect to the wired counterpart);
    \item Traffic patterns, user locations, and network
    conditions are constantly changing;
    \item Applications are heterogeneous with hard
    constraints required by the network;
    \item Energy and delay constraints change design
    principles across all layers of the stack.
\end{itemize}

\textbf{Multimedia requirements}
\vspace*{\fill}
\begin{center}
    \begin{tabular}{|c|c|c|c|c|}
        \hline
        & Voice & Data & Video & Game \\
        \hline
        Delay & low & irrelevant & low & low \\
        \hline
        Packet Loss & low & no & low & low \\
        \hline
        Bit Error Rate & $10^{-3}$ & $10^{-6}$ & $10^{-6}$ & $10^{-3}$ \\
        \hline
        Data Rate & 8-32 Kbps & 1-100 Mbps & 1-20 Mbps & 32-100 Kbps \\
        \hline
        Traffic & Continuous & Bursty & Continuous & Continuous \\
        \hline
    \end{tabular}
\end{center}
\vspace*{\fill}
\vspace*{\fill}
One-size-fits-all protocols and design
\begin{itemize}
    \item are used by wired networks $\rightarrow$ poor results;
    \item do not work well $\rightarrow$ Crosslayer design.
\end{itemize} 
\vspace*{\fill}
\newpage
\textbf{Crosslayer Design}

It's made of 5 layers:

\begin{minipage}{.2 \linewidth}
    \begin{tabular}{|c|}
        \hline
        Application\\
        \hline
        Network\\
        \hline
        Access\\
        \hline
        Link\\
        \hline
        Hardware\\
        \hline
    \end{tabular}
\end{minipage}
\begin{minipage}{.2 \linewidth}
    \begin{tabular}{l}
        $\rightarrow$ Meet delay, rate and energy constraints\\
        $\rightarrow$ Adapt across design layers\\
        $\rightarrow$ Reduce uncertainty through scheduling\\
        $\rightarrow$ Provide robustness via diversity\\
        \\
    \end{tabular}
\end{minipage}

\subsection{Wireless Systems}
There are different types of current wireless systems:
\begin{itemize}
    \item Cellular Systems;
    \item Wireless LANs;
    \item Satellite Systems;
    \item Bluetooth;
    \item \dots
\end{itemize}
And others which are emerging:
\begin{itemize}
    \item Ad hoc Wireless Network;
    \item Mesh Network;
    \item Sensor Network;
    \item Distributed Control Network;
    \item MANET/VANET/FANET;
    \item Underwater Networks;
    \item RFID;
    \item Nano-networks;
    \item \dots
\end{itemize}

\subsubsection{Cellular Systems}
Characteristics:
\begin{itemize}
    \setlength\itemsep{0.7em}
    \item every geographic region is divided into cells
    \begin{itemize}
        \item [$\rightarrow$] more transmission distance $\Rightarrow$
        more power;
    \end{itemize}
    \item frequency/timeslots/codes are reused at separated locations;
    \item co-channels interference between same color cells;
    \item base stations has control of functions and handoff;
    \item it can be shrinked to increase capacity and
    relax networking burden.
    \item it supports both voice (continuos) and data (bursty) requiring
    different:
    \begin{itemize}
        \item[$\rightarrow$] access
        \item[$\rightarrow$] routing strategies 
    \end{itemize}
    \item About connectivity:
    \begin{itemize}
        \item[$\rightarrow$] 3G: packet-based switching for both voice
        and data (up to 7.2 Mbps)
        \item[$\rightarrow$] 4G - 5G: are more focused on data
        (high bandwidth, high reliability, low latency)
    \end{itemize}
\end{itemize}

\subsubsection{Wireless Local Area Networks (WLANs)}

Characteristics:
\begin{itemize}
    \item devices are connected (wireless) to an AP\footAP\\
    $\rightarrow$ it is wired-connected to internet;
    \item breaks data into packets ($\approx$ 1500 B) $\rightarrow$ AP\footAP
    in even smaller size (500 B);
    \item MAC layer control access to shared channel (random access);
    \item backbone internet provides best-effort service
    \begin{itemize}
        \item[$\rightarrow$] bandwidth cannot be determined!
        \item[$\rightarrow$] users pay subscription only for home-access
        provider distance\\
        $\Rightarrow$ it can be bottleneck if the backbone is faster
        \item[$\rightarrow$] having QoS (subscription) here can increase digital gap
        \vspace{0.1cm}
        \item[] Server $\rightarrow$ Internet $\rightarrow$ Access Provider $\rightarrow$
        Access Point $\rightarrow$
        $\begin{cases}
            \text{device1}&\\
            \text{device2}&\\
            \text{\dots}&\\
            \text{deviceN}& 
        \end{cases}$
    \end{itemize}
\end{itemize}

There are different versions (802.11):
\begin{itemize}
    \item b (old gen): only 2.4 GHz, speed 1-11 Mbps, range 100m
    \item g (legacy std): 2.4-5 GHz, speed up to 54 Mbps 
    \item n (current gen): 2.4-5 GHz, speed up to 300 Mbps, multiple I/O
    \item ac (emerging gen): 2.4-5 GHz, speed up to 500 Mbps, multiple I/O
    \item s: used for mesh networks
    \item p: used for vehicular networks
\end{itemize}
\newpage
\subsubsection{Satellite Systems}
Satellites haven't been used so much until starlink which is gaining popularity because,
even if they make light pollution, they are very lightweight and easy to wake up.
There are many types of satellites:
\begin{itemize}
    \item GEO (Geostationary Earth Orbit);
    \item MEO (Medium Earth Orbit);
    \item LEO (Low Earth Orbit).
\end{itemize}
In particular satellites:
\begin{itemize}
    \item can cover large areas depending on their height in the space:
    \begin{itemize}
        \item[$\rightarrow$] $>$ height $\Rightarrow$ $>$ covered area, $>$ latency, $<$ bandwidth
        \item[$\rightarrow$] $<$ height $\Rightarrow$ $<$ covered area, $<$ latency, $>$ bandwidth
    \end{itemize}
    \item for one-way transmission are optimised (i.e. radio and movie broadcasting);
    \item for two-way transmission are given up because of costs and few ambitions.
\end{itemize}

\subsubsection{Bluetooth}

Characteristics:
\begin{itemize}
    \item it is a low cost replacement for cables;
    \item it covers a short range up to 100m with multihop\\
    $\rightarrow$ it requires exponential energy as distance
    grows
    \item frequency 2.4 GHz
    \item 4 channels (3 for voice, 1 for data up to 700 Kbps)
    \item Widely supported by telecommunications, PC\dots\\
    $\rightarrow$ it is a standard de facto (also BLE\dots)
\end{itemize}

\subsubsection{Ad Hoc Networks}

Characteristics:
\begin{itemize}
    \item it is a peer-to-peer communications (born for military purposes)
    \item there isn't any backbone infrastructure
    \item routing is very hard because of:
    \begin{itemize}
        \item[$\rightarrow$] dynamic topology;
        \item[$\rightarrow$] typically multihop $\rightarrow$ to
        extend coverage area or reduce interferences
    \end{itemize}
\end{itemize}

Problems:
\begin{multicols}{2}
\begin{itemize}
    \item hops;
    \item bandwidth;
    \item collsions handling;
\end{itemize}
\begin{itemize}
    \item energy consumption;
    \item topology;
    \item dependency on device.
\end{itemize}
\end{multicols}

\subsubsection{Mesh Networks}

Characteristics:
\begin{itemize}
    \item Ad hoc opportunistic extension of a fixed urban infrastructure\\
    $\rightarrow$ full of wireless acess point which can connect to other
    ones
    \item it is easier than ANET because of almost static topology;
    \item creation of wireless coverage which is:
    \begin{itemize}
        \item[$\rightarrow$] low-cost
        \item[$\rightarrow$] easily deployable
        \item[$\rightarrow$] high performancing
    \end{itemize}
    \item Challenges to face:
    \begin{itemize}
        \item[$\rightarrow$] QoS
        \item[$\rightarrow$] routing protocols optimisation for fairness and load balancing
        \item[$\rightarrow$] automatic setup on infrastructure's failures
    \end{itemize}
\end{itemize}

\subsubsection{Sensor Networks}

Characteristics:
\begin{itemize}
    \item there is at least one sensor as device in the network;
    \item energy is the principal constraint (low or no battery)
    \item data flows to centralised locations;
    \item low per-node rate $\rightarrow$ up to 100K nodes
    and they can cooperate in:
    \begin{itemize}
        \item[$\star$] transmission
        \item[$\star$] reception
        \item[$\star$] compression
        \item[$\star$] signal processing
    \end{itemize}
\end{itemize}

\subsubsection{Distributed Control over Wireless Links}

Characteristics:
\begin{itemize}
    \item it is a possibile scenario where there is contorl over something;
    \item it has to be robust to failures;
    \item Packet loss and delays impact controller performance;
    \item used mainly on autmated vehicles such as cars, UAVs\dots
\end{itemize}
\newpage
\subsubsection{Mobile Ad Hoc Networks (MANET)}

Characteristics:
\begin{itemize}
    \item ANET with a dynamic topology using:
    \begin{itemize}
        \item[$\rightarrow$] Infrastructure Network (WiFi or 3G/4G)
        \item[$\rightarrow$] Ad Hoc Multihop wireless Network
    \end{itemize}
    \item Instantly deployable and re-configurable (for temporary needs);
    \item Portable (i.e. sensors) and mobile (i.e. cars);
\end{itemize}

\subsubsection{Opportunistic Ad Hoc Networks}

Characteristics:
\begin{itemize}
    \item they are created when needed;
    \item Driven by “commercial” application needs:
    \begin{itemize}
        \item[$\rightarrow$] Indoor WLAN extended coverage
        \item[$\rightarrow$] Bluetooth sharing
        \item[$\rightarrow$] Peer-to-Peer networking on vehicles
    \end{itemize}
    \item Access to internet available\\ $\rightarrow$ BUT if too costly or inadequate
    $\Rightarrow$ replacement with Ad Hoc Network
\end{itemize}

\subsubsection{Vehicular Ad Hoc Networks (VANET)}

Characteristics:
\begin{itemize}
    \item ANET for vehicles
    \item it has 1000m range
    \item it supports 5.9 GHz
    \item it has 6-27 Mbps data rate depending on range
    \item it is more predictable $\rightarrow$ it may deduce infos
    $\Rightarrow$ useful for crosslayers
\end{itemize}

\subsubsection{Flying Ad Hoc Networks (FANET)}

Characteristics:
\begin{itemize}
    \item ANET for flying objects (i.e drone, mixed vehicles\dots)
    \item there is a 3D topology $\rightarrow$ protocols needs to be redesigned
\end{itemize}

\subsubsection{Underwater Sensor Networks}

Characteristics:
\begin{itemize}
    \item communication happens by sound $\rightarrow$
    messages propagate in circles;
    \item important to compute when message arrives $\rightarrow$ avoid collisions.
\end{itemize}

\subsubsection{Radio Frequecy IDentification (RFID)}

Characteristics:
\begin{itemize}
    \item it is based on tags (low cost), readers (high cost) and eventually
    a server;
    \item tags can have:
    \begin{itemize}
        \item[$\star$] no battery $\rightarrow$ emitter charges the tag with energy (steal control,\dots)
        \item[$\star$] battery $\rightarrow$ tag periodically emits its ID (check of product history, control with sensors,\dots)
    \end{itemize}
    \item systems can be built:
    \begin{itemize}
        \item[$\rightarrow$] lot of tags + one emitter $\Rightarrow$ cheap
        \item[$\rightarrow$] lot of emitters + one tag $\Rightarrow$ expensive
    \end{itemize}
    \item it can identify specific instance of a product! (not only type like barcode)
\end{itemize}