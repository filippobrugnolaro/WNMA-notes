\section{Vehicular Ad Hoc Networks (VANET)}
\subsection{IEEE 802.11p}
\begin{itemize}
    \item it provides connectivity to vehicles
    \begin{itemize}
        \item[$\rightarrow$] beyond 4G $\rightarrow$ not so crucial
        \item[$\rightarrow$] useful for safety/public apps 
    \end{itemize}
    \item parts taken from a and b versions
    \item it use a reserved sequency (5.9GHz)
    \begin{itemize}
        \item 1000 m of transmission range
        \item 26 Mbps $\rightarrow$ but it can be lower $\rightarrow$ sufficient for
        exchanging messages, \dots
        \item transmission up to 6 Mbps at 300m $\rightarrow$ nodes travelling at
        200 km/h speed
    \end{itemize}
    \item Problem:\\[0.15cm]
    connection with AP and vehicles
    \begin{itemize}
        \item[$\rightarrow$] transmission range changes
        \item[$\rightarrow$] transmission needs to be adapted
        \item[$\rightarrow$] these can lead to sync problems among vehicles
    \end{itemize}
\end{itemize}
\subsection{Vehicular Networks: System Model}
Introducing the context
\begin{itemize}
    \item Safe driving:
    \begin{itemize}
        \item[$\rightarrow$] alert messages $\rightarrow$ delivered quickly to all
        cars following the one that is alerting (Abnormal Vehicles)\\
        $\rightarrow$ need to generate a chain reaction
        \item[$\rightarrow$] there are some problems:
        \begin{itemize}
            \item multiple transmission $\Rightarrow$ when every node is broadcasting
            \item possible congestion $\Rightarrow$ alert unuseful $\rightarrow$ crashes
            still can happen
        \end{itemize}
    \end{itemize}
\end{itemize}
For these reason, a system model with these characteristics is introduced:
\begin{itemize}
    \item high mobility of nodes
    \item variable transmission range
    \item a car cannot be sure to be the farthest car receiving that broadcast message
    \item some approaches:
    \begin{itemize}
        \item[$\rightarrow$] MCDS (Minimum Connected Dominant Set):
        \begin{itemize}
            \item minimum cardinality set of connected nodes $\Rightarrow$ each
            other node in the network is connected to a node of the MCDS set
            \item MCDS nodes have to broadcast the message
            \item it is optimal but non feasible solution $\rightarrow$
            it needs creation of overlay\\structure $\Rightarrow$ to cover the network
            \item it is dangerous because of deterministic failure
            $\rightarrow$ what if a node in MCDS doesn't broadcast? $\rightarrow$ redundancy needed
            redundancy needed
            \item implementation with n nodes $\Rightarrow$ $O(n \cdot
            log(n))$ control messages
        \end{itemize}
        \item[$\rightarrow$] RD (Redundancy Avoidance):
        \begin{itemize}
            \item based on backoff mechanism $\rightarrow$ if there is a
            collision due congestion $\Rightarrow$ it reduces frequency
            \item if following vehicle has already broadcast $\Rightarrow$ actual
            nodes don't
            \item it isn't considered the number of hops in these schemas
        \end{itemize}
        \item[$\rightarrow$] JS (Jamming Signal):
        \begin{itemize}
            \item it is Urban Multi-hop Broadcasting Protocol
            \item it is used to determine next forwarder
            \item vehicles which receve alert $\rightarrow$ it emits JS for
            an amount of time\\ $\Rightarrow$ proportional to the distance from sender
            \item The last vehicle stopping the JS knows it is the last one\\
            $\rightarrow$ it forwards the alert message
            \item JS phase delays the transmission of the message\\
            $\Rightarrow$ not suitable for alert messages
        \end{itemize}
        \item[$\rightarrow$] CW (Contention Window):
        \begin{itemize}
            \item vehicles set CW inversely proportional to distance from sender
            \item this implies:
            \begin{itemize}
                \item no control traffic
                \item something unrealistic $\Rightarrow$ transmission rate not known\\
                $\rightarrow$ it needs to be determined in order to find the last one
            \end{itemize}
        \end{itemize}
    \end{itemize}
\end{itemize}

\subsection{Fast Broadcasting}
Characteristics:
\begin{itemize}
    \item fast broadcast is a solution designed to have alert messages covering
    the area of interest in as less time as possible (as few hops as possible)
    \item there are two types:
    \begin{itemize}
        \item[$\rightarrow$] probabilistic:
        \begin{itemize}
            \item Pro $\rightarrow$ reliable
            \item Cons $\rightarrow$ End-to-End delay
        \end{itemize}
        \item[$\rightarrow$] deterministic:
        \begin{itemize}
            \item Pro $\rightarrow$ End-to-End delay
            \item Cons $\rightarrow$ reliable
        \end{itemize} 
    \end{itemize}
    \item how it works:
    \begin{itemize}
        \item[$\rightarrow$] there are two phases:
        \begin{itemize}
            \item estimation phase $\rightarrow$ vehicles exchange hello
            messages to collect info $\rightarrow$ in order to estimate
            their own transmission range.
            \item broadcasting phase $\rightarrow$ transmission range estimation used
            to\\forward asap alert message to destination
        \end{itemize}
        \item[$\rightarrow$] here there is a detailed description of the two phases
        \begin{enumerate}
            \item estimation phase:
            \begin{itemize}
                \item continuosly run
                \item time is splitted into rounds
                \item one hello messages randomly sent every time round
                \item hello contains
                \begin{itemize}
                    \item sender's position
                    \item maximum frontward distance from which another vehicle\\
                    has been heard transmitting an hello message
                \end{itemize}
                \item About messages:
                \begin{itemize}
                    \item two types:
                    \begin{enumerate}
                        \item[$\star$] hello $\rightarrow$ information to estimate transmission range
                        \item[$\star$] alert $\rightarrow$ sender's transmission range
                    \end{enumerate}
                    \item variables:
                    \begin{enumerate}
                        \item[$\star$] CMBR (Current Maximum Backward Range)\\[0.15cm]
                        $\rightarrow$ computed by hearing hello from the back
                        $\Rightarrow$ someone could hear
                        \item[$\star$] CMFR (Current Maximum Frontward Range)\\[0.15cm]
                        $\rightarrow$ computed by hearing hello from the front
                        $\Rightarrow$ I could hear
                    \end{enumerate}
                \end{itemize}
            \end{itemize}
            \item broadcast phase:
            \begin{itemize}
                \item alert is generated by an Abnormal Vehicle
                \item alert is sent in broadcast to warn following vehicles
                \item alert includes estimated transmission range for that hop
                \item node receiving alert waits an amount of time\\
                $\rightarrow$ proportional  to the node's position with respect to\\the
                estimated maximum transmission range\\
                ($\rightarrow$ near $\Rightarrow$ more time)
                \item CW calculated as follow with R = range and D = distance:
                \begin{center}
                    $CW = \floor{(\frac{R_{max}-D}{R_{max}} \cdot (CW_{max}-CW_{min}))+CW_{min}}$
                \end{center}
                $\rightarrow$ if another car (farther from source) already forwarded\\
                $\Rightarrow$ the other doesn't forward\\[0.15cm]
                $\rightarrow$ wrong estimation $\Rightarrow$ possibly huge delay\\[0.15cm]
                $\rightarrow$ dynamic transmission rate $\Rightarrow$ CW can be lower
                for estimation
            \end{itemize}
        \end{enumerate}
    \end{itemize}
    \item ROFF (RObust Fast Forwarding):
    \begin{itemize}
        \item it is a multi-hop deterministic delay based
        \item estimation $\rightarrow$ every vehicle sends hello every round\\
        $\rightarrow$ neighbourhood discovery
    \end{itemize}
\end{itemize}

