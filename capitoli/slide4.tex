\section{MAC Layer}

\subsection{Introduction}
Multiple Access Control (MAC) layer:
\begin{itemize}
    \item it is a media access control protocol in which there is:
    \begin{itemize}
        \item[$\rightarrow$] coordination and scheduling of transmissions
        \item[$\rightarrow$] hosts competing for having the channel
    \end{itemize}
    \item Access control
    \begin{itemize}
        \item[$\rightarrow$] it is referred to shared channel
        \item[$\rightarrow$] broadcast of wireless transmission (at the light of speed)
        \item[$\rightarrow$] who can transmit when/where
        \item[$\rightarrow$] collisions $\rightarrow$ avoid/recover from them with detection or not\\
        $\Rightarrow$ the problem is receiving at the same time (NOT SENDING)
    \end{itemize}
    \item Goals:
    \begin{itemize}
        \item[$\rightarrow$] low latency
        \item[$\rightarrow$] good channel utilization (no collisions $\rightarrow$ using it as much as possible)
        \item[$\rightarrow$]  best effort + real time support
    \end{itemize}
    As in a human conversation:
    \begin{itemize}
        \item[$\rightarrow$] Everybody should have the chance to talk
        \item[$\rightarrow$] Do not speak until it is your turn
        \item[$\rightarrow$] Do not monopolize the conversation
        \item[$\rightarrow$] Raise your hand if you have to ask for something
        \item[$\rightarrow$] Do not interrupt while somebody is talking
        \item[$\rightarrow$] Do not fall asleep while somebody is talking
    \end{itemize}
    So the most important concepts are:
    \begin{itemize}
        \item[$\rightarrow$] efficiency in the bandwidth use $\rightarrow$ the maximum possible
        \item[$\rightarrow$] resilience $\rightarrow$ avoid collisions
        \item[$\rightarrow$] fairness $\rightarrow$ given n nodes and a bandwidth b, each one should have a bandwidth $\text{b}_{\text{n}}=\frac{\text{b}_{\text{tot}}}{\text{n}}$
        \item[$\rightarrow$] robustness $\rightarrow$ decentralised, no single point of failure
        \item[$\rightarrow$] simplicity $\rightarrow$ easy to implement
    \end{itemize}
    \item Channel Access Problem
    \begin{itemize}
        \item[$\rightarrow$] there is a multiple nodes share channel\\
        $\Rightarrow$ simultaneous communication is not possible
        \newpage
        \item[$\rightarrow$] MAC proocols give schemes to schedule communication
        \begin{itemize}
            \item maximise number of communication $\rightarrow$ avoid collisions
            \item guarantee fairness among all transmitters
        \end{itemize}
        \item[$\rightarrow$] trivial solution is Transmit and Pray\\$\Rightarrow$ plenty of collisions $\rightarrow$ poor throughput at high
        \item[$\rightarrow$] Carrier Sense Multiple Access (CSMA):
        \begin{itemize}
            \item it provides a fix to Transmit and Pray
            \item transmitters listen to the channel before sending $\rightarrow$ waiting when signal on channel
            \item collisions:
            \begin{itemize}
                \item can still occur due to propagation delay
                \item when it happens the entire packet could be lost $\rightarrow$ time wasted
            \end{itemize}
        \end{itemize}
    \end{itemize}
\end{itemize}

\subsection{MAC Protocols}
MAC protocol $\rightarrow$ coordinates transmissions from
different stations\\ \hspace*{2.5cm}$\Rightarrow$ minimize or avoid collisions\\[0.2cm]
There a 3 different types of protocols:
\begin{itemize}
    \item Channel partitioning (TDMA, FDMA, CDMA)
    \item Random Access (CSMA, MACA)
    \item Taking turns (polling)
\end{itemize}
Approaches to MAC layer are:
\begin{itemize}
    \item Random Access:
    \begin{itemize}
        \item[$\rightarrow$] Without carrier sensing $\rightarrow$ Pure Aloha, Slotted Aloha
        \item[$\rightarrow$] With carrier sensing $\rightarrow$ CSMA, CSMA/CD, MACAW
    \end{itemize}
    \item Controlled Access:
    \begin{itemize}
        \item[$\rightarrow$] Centralized $\rightarrow$ entity regulate channel's access (FDMA, TDMA, CDMA)
        \item[$\rightarrow$] Distributed $\rightarrow$ distributed apps with peer nodes regulate channel's\\ access (Token ring)
    \end{itemize}
\end{itemize}
\textbf{Random Access Protocols}

Characteristics:
\begin{itemize}
    \item node transmits at random at full channel data rate
    \item if nodes collide then they retransmit at random times
    \item each one dectects/recovers form collision in a different way
\end{itemize}
\vspace*{0.5cm}
Here there is the description of the most important protocols.

\subsubsection{Slotted Aloha}

Characteristics:
\begin{itemize}
    \item time is divided into equal size slots $\rightarrow$ equal to full packet size
    \item newly arriving station transmits at the beginning of the next slot
    \item if collision occurs:
    \begin{itemize}
        \item[$\rightarrow$] assumption of the presence of channel feedback
        \item[$\rightarrow$] retransmission of packet at each slot with probability P, until successful
        \end{itemize}
        \item Successful of transmission:\\[0.2cm]given:
        \begin{itemize}
            \item N = number of stations
            \item P = probability that each station transmits in the slot
            \item S = probability of successful of transmission
        \end{itemize}
        the value of S is:
        \begin{itemize}
            \item S $=$ p(1-p)$^{\text{(N-1)}}$ by a single node 
            \item S $=$ Np(1-p)$^{\text{(N-1)}}$ by any of N nodes
        \end{itemize}
    \item throughput efficiency is about $\frac{\text{1}}{\text{e}}$ $\rightarrow$
    and:
    \begin{enumerate}
        \item obtaining p = $\frac{\text{1}}{\text{N}}$ (p should be tailored based on N)
        \item substituting p to S $=$ Np(1-p)$^{\text{(n-1)}}$ $\Rightarrow$
        S $=$ N$\frac{\text{1}}{\text{N}}$(1-$\frac{\text{1}}{\text{N}}$)$^{\text{(N-1)}}$
        \item solving S at the limit obtaining S $=$ $\frac{\text{1}}{\text{e}}$
    \end{enumerate}
    \item it is fully decentralised
\end{itemize}
\subsubsection{Pure Aloha}
Characteristics:
\begin{itemize}
    \item it doesn't require time slots $\rightarrow$ no synchronization
    \item nodes can transmit at any time $\Rightarrow$ collision may increase
    \item Successful of transmission:\\[0.2cm]given:
    \begin{itemize}
        \item N = number of stations
        \item P = probability that each station transmits in the slot
        \item S = probability of successful of transmission
    \end{itemize}
    the value of S is:
    \begin{itemize}
        \item S $=$ p(1-p)$^{\text{2(N-1)}}$ by a single node 
        \item S $=$ Np(1-p)$^{\text{2(N-1)}}$ by any of N nodes
    \end{itemize}
    \item throughput efficiency is about $\frac{\text{1}}{\text{2e}}$ $\rightarrow$ every
    transmission can occupy 2 slots\\ $\rightarrow$ and:
    \begin{enumerate}
        \item obtaining p = $\frac{\text{1}}{\text{2(N-1)}}$ (p should be tailored based on N)
        \item substituting p to S $=$ Np(1-p)$^{\text{2(N-1)}}$ $\Rightarrow$
        S $=$ N$\frac{\text{1}}{\text{2(N-1)}}$(1-$\frac{\text{1}}{\text{2(N-1)}}$)$^{\text{2(N-1)}}$
        \item solving S at the limit obtaining S $=$ $\frac{\text{1}}{\text{2e}}$
    \end{enumerate}
\end{itemize}

\subsubsection{Considerations Pure \& Slotted Aloha}
Both are:
\begin{itemize}
    \item not efficient at all $\rightarrow$ a lot of retransmissions:
    \begin{itemize}
        \item[$\star$] Pure Aloha throughput $\rightarrow$ 18.4 \%
        \item[$\star$] Slotted Aloha throughput $\rightarrow$ 36.8 \%
    \end{itemize}
    \item unfair $\rightarrow$ aggressive senders can capture the channel
    \item robust $\rightarrow$ decentralized
    \item simple:
    \begin{itemize}
        \item[$\star$] Pure Aloha $\rightarrow$ no coordination
        \item[$\star$] Slotted Aloha $\rightarrow$ just synchronization
    \end{itemize}
\end{itemize}
\subsubsection{Carrier Sense Multiple Access (CSMA)}
Characteristics:
\begin{itemize}
    \item Aloha protocols are less performing $\rightarrow$ lack of coordination among nodes
    \item Each node continuously listens to channel $\rightarrow$ awareness of channel's freedom
    $\Rightarrow$ improve in efficiency
\end{itemize}
There are different types of CSMA:
\begin{itemize}
    \item 1-persistent CSMA
    \item non-persistent CSMA
    \item p-persistent CSMA
\end{itemize}

\paragraph{1-persistent CSMA}\mbox{}\\[0.2cm]
Characteristics:
\begin{itemize}
    \item how it works:
    \begin{enumerate}
        \item nodes listen to the channel
        \item the channel can be:
        \begin{itemize}
            \item[$\rightarrow$] free $\rightarrow$ immediate transmission
            \item[$\rightarrow$] busy $\rightarrow$ waiting until channel is free $\rightarrow$ $\text{P}_\text{R}$ $=$ 1\\\\\\
            $\Rightarrow$ P\textsubscript{R} $=$ probability of retransmission\\
            (if there is a collision $\rightarrow$ node waits for a random
            time and\\retries $\Rightarrow$ desynchronization)
        \end{itemize}
    \end{enumerate}
    \item propagation time
    \begin{itemize}
        \item[$\rightarrow$] impact on performance
        \item[$\rightarrow$] more time $\Rightarrow$ more collisions\\
        Example:\\
        A can't hear B $\rightarrow$ B is trasmitting for so much time and A want to transmit
        $\rightarrow$ channel is free for A but it is not $\Rightarrow$ collision
        \item[$\rightarrow$] even with no propagation time\\
        Example:\\if two nodes transmit and a third is occupying the channel\\
        $\rightarrow$ when channel is free $\rightarrow$ all 2 transmit at
        same time $\Rightarrow$ collision
    \end{itemize}
\end{itemize}

\paragraph{Non-persistent CSMA}\mbox{}\\[0.2cm]
Characteristics:
\begin{itemize}
    \item how it works:
    \begin{enumerate}
        \item nodes listen to the channel
        \item the channel can be:
        \begin{itemize}
            \item[$\rightarrow$] free $\rightarrow$ immediate transmission
            \item[$\rightarrow$] busy $\rightarrow$ waiting a random time and then retry to listen
        \end{itemize}
    \end{enumerate}
    \item it is less aggressive than 1-persistent CSMA
\end{itemize}

\paragraph{P-persistent CSMA}\mbox{}\\[0.2cm]
Characteristics:
\begin{itemize}
    \item it is slot based
    \item how it works:
    \begin{enumerate}
        \item nodes listen to the channel
        \item the channel can be:
        \begin{itemize}
            \item[$\rightarrow$] free $\rightarrow$ transmission with probability p
            \item[$\rightarrow$] busy $\rightarrow$ wait with probability (1-p) and then retry to listen
        \end{itemize}
    \end{enumerate}
    \item Aggressiveness:
    \begin{itemize}
        \item[$\rightarrow$] it depends on p
        \item[$\rightarrow$] p can be choose depending to the number of nodes:
        \begin{itemize}
            \item many $\rightarrow$ it may be conservative\\
            $\Rightarrow$ bandwidth waste depending on number of collisions
            \item few $\rightarrow$ it may be aggressive\\
            $\Rightarrow$ bandwidth waste depending on time of channel not used
        \end{itemize}
    \end{itemize}
\end{itemize}

\paragraph{CSMA with Collision Detection (CSMA/CD)}\mbox{}\\[0.2cm]
Characteristics:
\begin{itemize}
    \item it is like CSMA $\rightarrow$ but collisions are detected within few bit times
    \item when it is detected $\rightarrow$ transmission aborted $\Rightarrow$ reduction of channel wastage
    \item transmission is typically implemented persistently
    \item collision detection can approach channel utilization $=$ 1 in LANs\\ $\rightarrow$ it can detect immediately if something is wrong
    \item easy detection in wired LANs $\rightarrow$ it can
    measure signal strength\\ $\rightarrow$ on the line, or code violations, \dots
    \item collision detection can't be done in wireless LANs\\
    Example:\\
    the receiver shut off while transmitting $\rightarrow$ avoid damaging it with excess power
\end{itemize}













