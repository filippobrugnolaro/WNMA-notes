\section{Indoor Localization}
Localization $\rightarrow$ can provide a lot of useful services (best route, traffic jam,
\dots)\\It can be:
\begin{itemize}
    \item Outdoor $\rightarrow$  easy and eventually highly precise ($\approx$ 1 m).\\
    For example:
    \begin{itemize}
        \item[$\rightarrow$] GPS $\rightarrow$ satellites
        \item[$\rightarrow$] A-GPS $\rightarrow$ assisted, mobile cellular antennas
        helps to give a rough pos
    \end{itemize}
    \item Indoor $\rightarrow$ not easy (GPS signal not working)\\
    $\Rightarrow$ lots of applications can be built $\rightarrow$ with wide use of smartphones
\end{itemize}
Some notation:
\begin{itemize}
    \item Environment $\rightarrow$ assumed to have cartesian coordinates
    \item Mobile Station (MS) $\rightarrow$ device to be localized
    \item Base Station (BS) $\rightarrow$ infrastructure component such as Access Point \dots
\end{itemize}
Here there are some characteristics which describe better inddor localization

\subsection{Metrics}
Some metrics are:
\begin{itemize}
    \item Accuracy $\rightarrow$ average error between estimated
    and actual measure\\
    $\Rightarrow$ how much difference there is between measures
    \item Precision $\rightarrow$ error distribution of actual position
    vs the estimated one\\
    $\Rightarrow$ how measures are spread
    \item Robustness $\rightarrow$ ability in maintaining accurate
    estimation\\$\rightarrow$ even when chaging the context/environment
    \item Scalability $\rightarrow$ system behavior when changing 
    number/density devices
    \item Cost $\rightarrow$ includes hardware, initial set up, maintenance
    \dots
\end{itemize}

\subsection{Approaches}
\begin{itemize}
    \item Triangulation:
    \begin{itemize}
        \item[$\rightarrow$] it requires knowledge about arrival angles of signal\\
        emitted by MS $\Rightarrow$ received by BS
        \item[$\rightarrow$] characteristics:
        \begin{itemize}
            \item at least 2 angles are needed
            \item it requires complex hardware on BS
            \item indoor signal may be disturbed (walls \dots)
            \item not really usable for strong multipath effects
        \end{itemize}
    \end{itemize}
    \item Trilateration:
    \begin{itemize}
        \item[$\rightarrow$] it requires knowledge of the distance between MS and BS
        \item[$\rightarrow$] 2D $\Rightarrow$ 3BS, 3D $\Rightarrow$ 4 BS
    \end{itemize}
\end{itemize}
\vspace*{0.3cm}
Distance Estimation is done $\rightarrow$ propagation
time of radio signal\\[0.15cm]
$\Rightarrow$ electromagnetic waves with speed of light $\Rightarrow$
$d = c \cdot t_{prop}$
\begin{itemize}
    \item Time Of Arrival (TOA):
    \begin{itemize}
        \item[$\rightarrow$] MS and BS need synchronization $\Rightarrow$ it is hard
        \item[$\rightarrow$] how it works:
        \begin{itemize}
            \item BS emits signal to MS $\rightarrow$ including time end trasmission (t$_\text{1}$)
            \item MS completes reception of signal at time (t$_\text{2}$)
            \item MS computes propagation time $\rightarrow$ $t_{prop} = t_2 - t_1$
        \end{itemize}
    \end{itemize}
    \item RTT (Round Trip Time):
    \begin{itemize}
        \item[$\rightarrow$] it doesn't require data exchange or clock synchronization
        \item[$\rightarrow$] it measures time for path MS $\rightarrow$ BS $\rightarrow$ MS
        \item[$\rightarrow$] so time propagation is $t_{prop} = \frac{t_{remote} - t_{local}}{2}$\\[0.15cm]
        $\rightarrow$ where:
        \begin{itemize}
            \item $t_{local}$ $\rightarrow$ depends on reaction time of hardware
            \item $t_{remote}$ $\rightarrow$ from data to ACK
        \end{itemize}
    \end{itemize}
\end{itemize}
\vspace*{0.3cm}
Measurement Error:
\begin{itemize} 
    \item it depends on granularity of timer useed to measure
    \item Example:\\[0.15cm]
    \hspace*{-0.05cm}802.11 MAC layer $\rightarrow$ hardware allows timestamp packet  1$\mu$s precision\\
    $\Rightarrow$ it is high for indoor buildings $\rightarrow$ granularity of 300 m at speed light\\
    $\rightarrow$ 1 ns could be OK
    \item precision is not sufficient so there are 2 approaches:
    \begin{itemize}
        \item[$\rightarrow$] Hardware:
        \begin{itemize}
            \item it uses timestamp provided by modified hardware
            \item it is used specific hardware to trigger MAC layer counter\\
            $\rightarrow$ based on WLAN board's clock
            \item $\approx 7$ m precision
            \item it can be competitionbined with software approaches
            \item measurements done in lowest possible layer (MAC layer)\\
            $\Rightarrow$ to avoid software delays of other layers
            \item RTT is measured with data ACK (from reception of data to SIFS)
        \end{itemize}
        \newpage
        \item[$\rightarrow$] Software:
        \begin{itemize}
            \item it uses multiple measurements to obtain estimation close to\\
            actual value
            \item timestamp is provided by regular WLAN boards\\
            $\rightarrow$ only for received packets
            \item need to introduce monitoring station:
            \begin{itemize}
                \item to monitor communications between MS/BS
                \item it is better if it is closer to MS
                \item it collects a lot of timestamp both for send/received packets\\
                $\rightarrow$ then estimates the distance 
            \end{itemize}
            \item Goodtry:
            \begin{itemize}
                \item developed for research purposes
                \item it measures RTS - CTS - DATA - ACK sequences\\[0.15cm]
                $\rightarrow$ measuring twice the RTT for each transmission
                \item positioning is done through the lowest weigth squared error\\[0.15cm]
                $\rightarrow$ management of multiple measurements
            \end{itemize}
        \end{itemize}
    \end{itemize}
\end{itemize}
\subsection{Other Approaches}
Different approaches are:
\begin{itemize}
    \item TDOA (Time Difference Of Arrival)
    \begin{itemize}
        \item[$\rightarrow$] it measures arrival time of signal emitted by MS towards multiple BS
        \item[$\rightarrow$] it exploits differences among the arrival times to extract the MS position\\
        $\Rightarrow$ not suitable for self-positioning
        \item[$\rightarrow$] characteristics:
        \begin{itemize}
            \item sync needed for BSs $\rightarrow$ 2D $\Rightarrow$ 3BS, 3D $\Rightarrow$ 4BS
            \item server needed to manage both sync and measurements collection
        \end{itemize} 
    \end{itemize}
    \item Scene Analysis:
    \begin{itemize}
        \item[$\rightarrow$] method composed by 2 phases:
        \begin{enumerate} 
            \item fingerprint collection $\rightarrow$ start to collect fingerprints of known location of
            different BSs $\Rightarrow$ fingerprint $=$ signal strength and other parameters
            \item actual evaluation $\rightarrow$ it compares actual values with the stored ones\\
            $\rightarrow$ with AI algorithms (SVM, k-NN \dots) or statistical methods
        \end{enumerate}
        \item[$\rightarrow$] it is not robust $\rightarrow$ because new room $\Rightarrow$ new fingerprint to be collected\\
        $\Rightarrow$ big initial effort and offline training
        \item[$\rightarrow$] 2D $\Rightarrow$ 3BS, 3D $\Rightarrow$ 4BS
        \item[$\rightarrow$] RSS (Recieved Signal Strength) depends on:
        \begin{itemize}
            \item Multipath $\rightarrow$ measured strength is higher than ideal\\
            $\Rightarrow$ because of reflected signals
            \item Shadowing $\rightarrow$ in case of NLOS (non-light-of-sight) $\Rightarrow$
            signal can't be easily computed $\Rightarrow$ signal absorbed
            \item Moving Objects $\rightarrow$ cause high obscillations of RSS\\$\rightarrow$
            need multiple measurements
        \end{itemize}
        \item[$\rightarrow$] k-NN (k-Nearest Neighbour) algorithm:
        \begin{itemize}
            \item Given this notation first:
            \begin{itemize}
                \item m $=$ number of BS's
                \item n $=$ number of fingerprints in the training set
                \item S$_i$ $=$ fingerprint corresponding to point $(x_i$, $y_i$, $z_i)$ of training set
                \item s $=$ Measurement of RSS performed by MS in online phase          
            \end{itemize}
            \item it computes distance $s$ from every fingerprint in the training set
            \item k points are chosen in the training set with smallest d$_i^2$ values
            \item MS coordiantes $\rightarrow$ estimated as either:
            \begin{itemize}
                \item mean of coordinates of k locations
                \item weighted mean of distances
            \end{itemize} 
        \end{itemize}
    \end{itemize}
    \item RFID (Radio Frequency Identification) $\rightarrow$ system composed of:
    \begin{itemize}
        \item[$\rightarrow$] RFID reader:
        \begin{itemize}
            \item it emits a signal to query tags which are in proximity
            \item it receives the ID of each tag as a response
            \item it is the most expensive
        \end{itemize}
        \item[$\rightarrow$] RFID tag:
        \begin{itemize}
            \item it answers reader's queries with its ID $\rightarrow$ the way it is done can be:
            \begin{itemize}
                \item passive $\rightarrow$ cheaper, short range, long life battery
                \item active $\rightarrow$ expensive, long range, short life battery
            \end{itemize}
        \end{itemize}
    \end{itemize}
    \item Active RFID example (Landmark):
    \begin{itemize}
        \item[$\rightarrow$] it is a positioning system that exploits active RFID
        \item[$\rightarrow$] it uses scene analysis based on RSSI
        \item[$\rightarrow$] fingerprinter is considered as an array of RSSI\\
        $\Rightarrow$ emitted by tag and received by MS
        \item[$\rightarrow$] it is composed of:
        \begin{itemize}
            \item RFID readers $\rightarrow$ used as BS, communicate with localization servers
            \item Reference tag $\rightarrow$ tag with known coordinates
            \item Tracking tag $\rightarrow$ tag to be localized (MS)
        \end{itemize}
        \item[$\rightarrow$] it is robust because:
        \begin{itemize}
            \item reference tag doesn't require offline phase
            \item reference tag position can be dynamically measured
        \end{itemize} 
        \item[$\rightarrow$] for localization k-NN is used $\rightarrow$ so:
        \begin{itemize}
            \item it compares signal of tracking/reference tag
            \item system can be affected by the hardware\\[0.15cm]
            $\Rightarrow$ not all RFID readers provide sufficiently fine granularity of RSS\\[0.15cm]
            $\rightarrow$ active RFID tags are powered by a battery
            $\rightarrow$ system requires that\\transmission power of all tags has to be similar\\[0.15cm]
            $\Rightarrow$ need to use RFID tags of the same type and with the same level of battery to make a comparison
        \end{itemize}
    \end{itemize}
    \item Passive RFID:
    \begin{itemize}
        \item[$\rightarrow$] it is ok and less expensive when there is:
        \begin{itemize}
            \item wide space
            \item need to lacalize few nodes
        \end{itemize}
        \item[$\rightarrow$] Pro:
        \begin{itemize}
            \item less expensive
            \item ok for automated environments $\rightarrow$ predictable and low mobility
        \end{itemize}
        \item[$\rightarrow$] Cons:
        \begin{itemize}
            \item training phase is expensive\\$\rightarrow$ need of more snapshots ($\Rightarrow$ $\neq$ measures of same location)
            \item not to robust to environment changes
        \end{itemize}
    \end{itemize}
\end{itemize}

\subsection{Augmented Reality (AR)}
Characteristics:
\begin{itemize}
    \item it is based on the superimposition of informative
    levels to the real world\\(virtual, multimedia, geolocalized elements, \dots)
    \item it uses GPS, compass, accelerometer to adjust position/informations
    \item it is used position extraction through artificial markers (there is ARToolKit)\\
    $\rightarrow$ need to know position of the markers in the camera
    \item Coordinates translation (basics):
    \begin{itemize}
        \item[$\rightarrow$] position of object in space is represented by a matrix M with some rotation/translation submatrices
        \item[$\rightarrow$] given a point $P$, to extract cordinates of a point $P'$\\
        $\Rightarrow$ M $\cdot$ (matrix with coordinates of P)
        \item[$\rightarrow$] ARToolKit provides:
        \begin{itemize}
            \item $P_m$ $=$ position of marker in camera reference system
            \item $P_c$ $=$ position of camera in marker reference system $\Rightarrow$ $P_c = P_m^{-1}$
            \item from $P_c$ it is possible to extract translation array $T_c$
            \item it is possible to derive the marker global position thanks to the\\
            translation matrix $T_c$
        \end{itemize}
    \end{itemize}
    \item Refernce Systems:
    \begin{itemize}
        \item[$\rightarrow$] with a tag $\rightarrow$ it is possible to superimpose the object\\
        $\rightarrow$ determine its position is determined respect of what you're looking at
        \item[$\rightarrow$] with a camera $\rightarrow$ try to determine camera position respect of tag by using a picture
        \item[$\rightarrow$] your position $\Rightarrow$ tag position $\rightarrow$ if you find a tag $\Rightarrow$ you can be localized
        \item[$\rightarrow$] coverage of system $\Rightarrow$ 5 MP images can recognize a 20x20 cm marker at 11 m of distance
        \item[$\rightarrow$] error sources can happen because:
        \begin{itemize}
            \item smartphone generates JPEG (quality loss)\\
            $\rightarrow$ min compression $\Rightarrow$ good results
            \item calibration helps to compensate optic errors ($\Rightarrow$ distorted images)
            \item position estimation error $\rightarrow$ proportional to due to the error in the orientation computation
        \end{itemize} 
    \end{itemize}
    \item Multimarker:
    \begin{itemize}
        \item[$\rightarrow$] reasons:
        \begin{itemize}
            \item experiments with single marker/tag $\Rightarrow$ too many errors
            \item if markers appear not frontally to camera $\Rightarrow$ less errors
        \end{itemize}
        \item[$\rightarrow$] it can be:
        \begin{itemize}
            \item planar
            \item octagonal
        \end{itemize}
        \item[$\rightarrow$] creation of Octagonal Multimarker algorithm
        \begin{itemize}
            \item it is made of markers on different planes
            \item estimations on camera position and pictures are made
            \item non-frontal markers are privileged
        \end{itemize}
        \item[$\rightarrow$] Results:
        \begin{itemize}
            \item 2 markers are enough
            \item average error is near 22cm
        \end{itemize}
        \item[$\rightarrow$] Advantages respect to radiowaves systems:
        \begin{itemize}
            \item high precision
            \item lower cost for the hardware
            \item lower installation/configuration time $\rightarrow$ < to
            scene analysis methods
        \end{itemize}
        \item[$\rightarrow$] Disadvantages respect to radiowaves systems:
        \begin{itemize}
            \item Line of sight should be free between smartphone and one marker
            \item Semi-automatic system
        \end{itemize}
    \end{itemize}
\end{itemize}